\section{Oscillatory drive problem}
\subsection{Cas simple}
On considère un hamiltonien dépendant du temps de la forme 

\begin{equation}
    H(t) = H_0 + V(t)
\end{equation}

où $H_0 = \sum_{k} \lambda_k \ket{k}\bra{k}$ dans sa base d'états propres et $V(t) = X\cos(\omega t)$ pour un certain opérateur $X$. En décomposant le cosinus en exponentielles complexes, 

\begin{equation*}
    H(t) = H_0 + X\left(\frac{e^{i \omega t} + e^{-i\omega t}}{2}\right)
\end{equation*}

Pour une variation infime de temps et en utilisant la définition de (2.1), on trouve (et ce n'est pas très beau)

\begin{equation*}
    U(t+\delta t, t) = \sum_{n=0}^{\infty} (-i)^n \int_{t}^{t+\delta t}\int_{t}^{t_n}...\int_{t}^{t_2}H(t_n)...H(t_1)dt_1 ... dt_n
\end{equation*}
\begin{equation*}
    = \sum_{n=0}^{\infty} (-i)^n \int_{t}^{t+\delta t}\int_{t}^{t_n}...\int_{t}^{t_2}\left(H_0 + 
    V(t_n)\right)...\left(H_0 + V(t_1)\right)dt_1 ... dt_n
\end{equation*}
\begin{equation*}
    = \sum_{n=0}^{\infty} (-i)^n \int_{t}^{t+\delta t}\int_{t}^{t_n}...\int_{t}^{t_2} \left(H_0...H_0 + H_0...H_0V(t_1) + ... + V(t_n)H_0...H_0 + ... + V(t_n)...V(t_1)\right)dt_1 ... dt_n
\end{equation*}
\begin{equation}
    = \sum_{n=0}^{\infty} (-i)^n \int_{t}^{t+\delta t}\int_{t}^{t_n}...\int_{t}^{t_2} H_0...H_0 dt_1...dt_n + ... + \sum_{n=0}^{\infty} (-i)^n \int_{t}^{t+\delta t}\int_{t}^{t_n}...\int_{t}^{t_2} V(t_n)...V(t_1) dt_1...dt_n
\end{equation}

On peut voir que la distribution des termes correspond à l'ensemble des combinaisons de $n$ opérateurs où on choisit soit $H_0$ ou $V(t)$ pour chacun d'eux. Il y a donc au total dans la longue parenthèse $2^n$ termes chacun $n$ opérateurs. Certains d'entre eux ont un seul $V(t)$ placé à différents endroits dans la chaîne de $n$ opérateurs, d'autres en ont 2, etc... 

Les $V(t)$, selon leur positionnement, viennent briser des suites de $H_0...H_0$ en plusieurs petites chaînes. Par exemple, un des termes dans (2.2) est 

\begin{equation*}
    H_0...H_0V(t_1) = (H_0)^{n-1}V(t_1)    
\end{equation*}

et on voit alors qu'il y a une chaîne de $n-1$ opérateurs $H_0$ à gauche de $V(t_1)$. Sinon, un des termes présents

\begin{equation*}
    H_0...H_0V(t_4)H_0H_0V(t_1) = (H_0)^{n-4}V(t_4)(H_0)^2V(t_1)    
\end{equation*}

nous dit qu'il y a d'abord $V(t_1)$, puis deux $H_0$, ensuite $V(t_4)$ et finalement $n-4$ opérateurs $H_0$. De plus, par définition, on peut remplacer les $V(t)$ par $X\left(\frac{e^{i \omega t} + e^{-i\omega t}}{2}\right)$. Ainsi,

\begin{equation*}
    (H_0)^{n-1}V(t_1) = (H_0)^{n-1}X\left(\frac{e^{i \omega t_1} + e^{-i\omega t_1}}{2}\right) = \frac{1}{2}\left(e^{i \omega t_1} + e^{-i\omega t_1}\right)(H_0)^{n-1}X
\end{equation*}
\begin{equation*}
    (H_0)^{n-4}V(t_4)(H_0)^2V(t_1) = \frac{1}{2^2}\left(e^{i \omega t_1} + e^{-i\omega t_1}\right)\left(e^{i \omega t_4} + e^{-i\omega t_4}\right)(H_0)^{n-4}X(H_0)^2X
\end{equation*}








