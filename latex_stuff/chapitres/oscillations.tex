\newpage
\section{Application à un drive oscillant}
\subsection{Préliminaires}
Un drive est un laser qu'on envoie sur un transmon afin de manipuler son état. Contrairement à un pulse qui ne dure qu'un bref instant, un drive est appliqué plus longtemps de manière à avoir une rentrée d'énergie prolongée. Ici, le drive oscille et son application occasionne au transmon des changements de niveaux d'énergie. Le qubit évolue donc avec un hamiltonien dépendant du temps. Dans un tel contexte, ce dernier correspond à  

\begin{equation}
    H(t) = H_0 + V(t)
\end{equation}

où, dans sa base d'états propres, $H_0 = \sum_{k} \lambda_k \ket{k}\bra{k}$ correspond aux énergies du transmon quand on le laisse tranquille et $V(t) = X\cos(\omega t)$, pour un opérateur quelconque $X$, correspond à la modification due au drive. En décomposant le cosinus en exponentielles complexes, 

\begin{equation*}
    H(t) = H_0 + X\left(\frac{e^{i \omega t} + e^{-i\omega t}}{2}\right)
\end{equation*}

On s'attarde à l'opérateur d'évolution dans cette situation. Pour une variation infime de temps et en utilisant (2.1), on trouve (et ça fait peur)

\begin{equation*}
    U(t+\delta t, t) = \sum_{n=0}^{\infty} (-i)^n \int_{t}^{t+\delta t}\int_{t}^{t_n}...\int_{t}^{t_2}H(t_n)...H(t_1)dt_1 ... dt_n
\end{equation*}
\begin{equation*}
    = \sum_{n=0}^{\infty} (-i)^n \int_{t}^{t+\delta t}\int_{t}^{t_n}...\int_{t}^{t_2}\left(H_0 + 
    V(t_n)\right)...\left(H_0 + V(t_1)\right)dt_1 ... dt_n
\end{equation*}
\begin{equation*}
    = \sum_{n=0}^{\infty} (-i)^n \int_{t}^{t+\delta t}\int_{t}^{t_n}...\int_{t}^{t_2} \left(H_0...H_0 + H_0...H_0V(t_1) + ... + V(t_n)H_0...H_0 + ... + V(t_n)...V(t_1)\right)dt_1 ... dt_n
\end{equation*}
\begin{equation}
    = \sum_{n=0}^{\infty} \int_{t}^{t+\delta t}\int_{t}^{t_n}...\int_{t}^{t_2} (-i)^n H_0...H_0 dt_1...dt_n + ... + \sum_{n=0}^{\infty} \int_{t}^{t+\delta t}\int_{t}^{t_n}...\int_{t}^{t_2}(-i)^n V(t_n)...V(t_1) dt_1...dt_n
\end{equation}

On peut voir que la distribution des termes correspond à l'ensemble des combinaisons de $n$ opérateurs où on choisit soit $H_0$ ou $V(t)$ pour chacun d'eux. Il y a donc au total $2^n$ termes, chacun de $n$ opérateurs et pouvant alterner entre des suites de $H_0$ ou de $V(t)$ de différentes longueurs. Graphiquement, on peut représenter la génération des combinaisons par la figure 2 où chacune d'entre elles correspond à une branche différente de l'arbre.

\begin{figure}[H]
    \centering
     \includegraphics[width=0.45\textwidth]{images/ch2/embranchements.png}
    \caption{Arbre permettant de générer l'ensemble des combinaisons}
\end{figure}

On s'apprête maintenant à réécrire (2.2) différemment et la dérivation pour y arriver peut facilement porter à confusion. On essaie d'expliquer le plus possible chaque étape.

\subsection{Dérivation (partie 1)}
On considère maintenant $n$ comme étant le nombre de $V(t)$ présents dans chaque terme de (2.2) et on omet temporairement les indices $V(t_j)$ pour éviter de se mélanger avec l'ancienne écriture. Il ne s'agit plus du même $n$ que dans (2.2) et il faut alors un nouveau moyen d'écrire tout cela avec ce changement de variables. Pour se faire, on introduit les $m_i$ qui indiquent combien d'applications de $H_0$ il y a avant une application d'un $V(t)$. Par exemple,

\begin{equation*}
    H_0V(t)H_0H_0 = (H_0)^1V(t)(H_0)^2 = (H_0)^{m_1}V(t)(H_0)^{m_0} \implies m_0 = 2, m_1 = 1
\end{equation*}
\begin{equation*}
    V(t)V(t) = (H_0)^0V(t)(H_0)^0V(t)(H_0)^0 \implies m_0 = m_1 = m_2 = 0
\end{equation*}

En général, ils sont indexés de $m_0$ à $m_n$, car pour un nombre $n$ de $V(t)$, on peut avoir jusqu'à $n+1$ blocs $H_0...H_0$ ayant des longueurs différentes.

\begin{equation*}
    \underline{H_0H_0}V(t) : \text{ 1 bloc, } \underline{H_0}V(t)\underline{H_0} : \text{ 2 blocs, } V(t)\underline{H_0H_0} : \text{ 1 bloc}
\end{equation*}



Cependant, chaque terme de (2.2) a une somme infinie. Ainsi, ces blocs peuvent être arbitrairement longs, faisant en sorte que les $m_i$ peuvent prendre des valeurs entre 0 et l'infini. Il est plus concis de les mettre dans un vecteur $\boldsymbol{m} = \left[m_n, ..., m_0\right] \in \mathbb{Z}^{n+1}_+$. Ainsi, on peut écrire toute chaîne d'opérateurs comme 

\begin{equation}
    (H_0)^{m_n}V(t)(H_0)^{m_{n-1}}V(t)...(H_0)^{m_1}V(t)(H_0)^{m_0}
\end{equation}

dont on obtient sa longueur $M$ (le nombre total d'opérateurs) grâce à 

\begin{equation}
    M = \left(\sum_{i=0}^{n}m_i\right) + n
\end{equation}

\subsection{Dérivation (partie 2)}
La prochaine étape est maintenant de retrouver les indices dans les $V(t)$ avec notre nouvelle notation. On sait déjà qu'il y aura $n$ opérateurs $V(t)$, mais on doit pouvoir retrouver leur positionnement dans la chaîne d'opérateurs. On peut simplement compter combien il y a de $H_0$ et d'autres $V(t)$ avant celui qui nous intéresse. Par exemple, pour $n=2$, on pourrait avoir la chaîne

\begin{equation*}
    H_0V(t)H_0V(t)H_0 \implies \boldsymbol{m} = \left[1, 1, 1\right]
\end{equation*}

Le premier $V(t)$ s'applique nécessairement après $m_0 = 1$ opérateur $H_0$. Le deuxième s'applique nécessairement après $m_0 = 1$ opérateur $H_0$, un $V(t)$ puis $m_1 = 1$ opérateur $H_0$. Si $p \in \left[1..n\right]$ est une variable qui passe au travers des $n$ opérateurs $V(t)$, alors il est facile d'avoir l'indexage $V(t_{l(p)})$ où

\begin{equation}
    l(p) = \left(\sum_{j=0}^{p-1}m_j\right) + p
\end{equation}

Pour reprendre l'exemple, 

\begin{equation*}
    H_0V(t_{l(2)})H_0V(t_{l(1)})H_0 = H_0V(t_{m_1 + m_0 + 2})H_0V(t_{m_0 + 1})H_0 = H_0V(t_4)H_0V(t_2)H_0
\end{equation*}

comme on aurait dans (2.2).

\subsection{Dérivation (partie 3)}
Ensuite, on étend les $V(t_{l(p)})$ selon leur définition qu'on rappelle ici.

\begin{equation*}
    V(t_{l(p)}) = X\left(\frac{e^{i\omega t_{l(p)}} + e^{-i\omega t_{l(p)}}}{2}\right)
\end{equation*}

Toujours avec le même exemple,

\begin{equation*}
    H_0V(t_4)H_0V(t_2)H_0 = \left(\frac{e^{i\omega t_4} + e^{-i\omega t_4}}{2}\right)\left(\frac{e^{i\omega t_2} + e^{-i\omega t_2}}{2}\right)H_0XH_0XH_0 
\end{equation*}
\begin{equation*}
    = \frac{1}{2^2}\left(e^{i\omega t_2}e^{i\omega t_4} + e^{i\omega t_2}e^{-i\omega t_4} + e^{-i\omega t_2}e^{i\omega t_4} + e^{-i\omega t_2}e^{-i\omega t_4}\right)H_0XH_0XH_0
\end{equation*}

On introduit maintenant le vecteur à $n$ dimensions $\boldsymbol{\omega}_n$ dont chacun de ses éléments peut soit être $+\omega$ ou $-\omega$. On peut aller chercher l'élément $i$ par $\boldsymbol{\omega}_n[i]$. Pour un $n$ donné, on voit qu'il en existe $2^n$ différents qu'on rassemble dans $\left\{\boldsymbol{\omega}_n\right\}$. Par exemple, 

\begin{equation*}
    \left\{\boldsymbol{\omega}_2\right\} = \left\{\left[+\omega, +\omega\right], \left[+\omega, -\omega\right], \left[-\omega, +\omega\right], \left[-\omega, -\omega\right]\right\}    
\end{equation*}

On peut alors réécrire le produit d'exponentielles comme

\begin{equation*}
    \frac{1}{2^2}\sum_{\left\{\boldsymbol{\omega}_2\right\}}\left(\prod_{p=1}^{2}e^{i\boldsymbol{\omega}_2[p]t_{l(p)}} H_0XH_0XH_0\right)
\end{equation*}

Ainsi, (2.3) peut être réécrit de la manière suivante.

\begin{equation}
    \frac{1}{2^n}\sum_{\left\{\boldsymbol{\omega}_n\right\}}\left(\prod_{p=1}^{n}e^{i\boldsymbol{\omega}_n[p]t_{l(p)}} (H_0)^{m_n}X ... X(H_0)^{m_0}\right)
\end{equation}

\subsection{Dérivation (partie 4)}
Il est maintenant temps de remettre les précédentes parties dans le contexte de (2.2). D'abord, on se retrouve avec

\begin{equation*}
    \int_{t}^{t + \delta t}\int_{t}^{t_M}...\int_{t}^{t_2} (-i)^M \left(\frac{1}{2^n}\sum_{\left\{\boldsymbol{\omega}_n\right\}}\left(\prod_{p=1}^{n} e^{i\boldsymbol{\omega}_n[p]t_{l(p)}} (H_0)^{m_n}X...X(H_0)^{m_0}\right)\right)dt_1 ... dt_M
\end{equation*}
\begin{equation*}
    = \frac{1}{2^n}\sum_{\left\{\boldsymbol{\omega}_n\right\}}\left(\int_{t}^{t + \delta t}\int_{t}^{t_M}...\int_{t}^{t_2} (-iH_0)^{m_n}X...X(-iH_0)^{m_0} \cdot (-i)^n \prod_{p=1}^{n}e^{i\boldsymbol{\omega}_n[p]t_{l(p)}} dt_1 ... dt_M\right)
\end{equation*}

Il y a ensuite une somme infinie pour chaque terme qu'on absorbe dans les différentes longueurs $m_i$. On doit donc ajouter

\begin{equation*}
    \frac{1}{2^n}\sum_{\left\{\boldsymbol{\omega}_n\right\}}\sum_{\boldsymbol{m} \in \mathbb{Z}^{n+1}_+}\left(\int_{t}^{t + \delta t}\int_{t}^{t_M}...\int_{t}^{t_2} (-iH_0)^{m_n}X...X(-iH_0)^{m_0} \cdot (-i)^n \prod_{p=1}^{n}e^{i\boldsymbol{\omega}_n[p]t_{l(p)}} dt_1 ... dt_M\right)
\end{equation*}

Puis, on doit sommer la précédente équation pour tous les nombres $n$ de $V(t)$.

\begin{equation*}
    U(t+\delta t, t) = \sum_{n=0}^{\infty}\sum_{\left\{\boldsymbol{\omega}_n\right\}}\frac{1}{2^n}\sum_{\boldsymbol{m} \in \mathbb{Z}^{n+1}_+}\left(\int_{t}^{t + \delta t}\int_{t}^{t_M}...\int_{t}^{t_2} (-iH_0)^{m_n}X...X(-iH_0)^{m_0} \cdot (-i)^n \prod_{p=1}^{n}e^{i\boldsymbol{\omega}_n[p]t_{l(p)}} dt_1 ... dt_M\right)
\end{equation*}

\subsection{Dérivation (dernière partie)}
Finalement, on procède simplement à un changement de variables $t_i^{'} = t_i - t$. Ainsi, 

\begin{equation*}
    dt_i^{'} = dt_i - dt = dt_i    
\end{equation*}
\begin{equation*}
    t_i = t_i^{'} + t
\end{equation*}
\begin{equation*}
    t_i \in \left[t, t_j\right] \implies t_i^{'} \in [0, t_j^{'}]
\end{equation*}

nous donnent tout ce qu'il faut pour faire le changement de variables.

\begin{equation*}
    U(t+\delta t, t) = \sum_{n=0}^{\infty}\sum_{\left\{\boldsymbol{\omega}_n\right\}}\frac{1}{2^n}\sum_{\boldsymbol{m} \in \mathbb{Z}^{n+1}_+}\left(\int_{0}^{\delta t}\int_{0}^{t_M^{'}}...\int_{0}^{t_2^{'}} (-iH_0)^{m_n}X...X(-iH_0)^{m_0} \cdot (-i)^n \prod_{p=1}^{n}e^{i\boldsymbol{\omega}_n[p](t_{l(p)}^{'} + t)} dt_1^{'} ... dt_M^{'}\right)
\end{equation*}
\begin{equation*}
    = \sum_{n=0}^{\infty}\sum_{\left\{\boldsymbol{\omega}_n\right\}}\left(\prod_{p=1}^{n}e^{i\boldsymbol{\omega}_n[p]t}\right)\frac{1}{2^n}\sum_{\boldsymbol{m} \in \mathbb{Z}^{n+1}_+}\left(\int_{0}^{\delta t}\int_{0}^{t_M^{'}}...\int_{0}^{t_2^{'}} (-iH_0)^{m_n}X...X(-iH_0)^{m_0} \cdot (-i)^n \prod_{p=1}^{n}e^{i\boldsymbol{\omega}_n[p]t_{l(p)}^{'}} dt_1^{'} ... dt_M^{'}\right)
\end{equation*}
\begin{equation*}
    = \sum_{n=0}^{\infty}\sum_{\left\{\boldsymbol{\omega}_n\right\}}e^{i\sum_{p=1}^{n}\boldsymbol{\omega}_n[p]t}\frac{1}{2^n}\sum_{\boldsymbol{m} \in \mathbb{Z}^{n+1}_+}\left(\int_{0}^{\delta t}\int_{0}^{t_M^{'}}...\int_{0}^{t_2^{'}} (-iH_0)^{m_n}X...X(-iH_0)^{m_0} \cdot (-i)^n \prod_{p=1}^{n}e^{i\boldsymbol{\omega}_n[p]t_{l(p)}^{'}} dt_1^{'} ... dt_M^{'}\right)
\end{equation*}

Ce n'est pas très beau, alors on définit

\begin{equation}
    S^{(n)}_{\boldsymbol{m}}(\boldsymbol{\omega}_n, \delta t) = \int_{0}^{\delta t}\int_{0}^{t_M^{'}}...\int_{0}^{t_2^{'}} (-iH_0)^{m_n}X...X(-iH_0)^{m_0} \cdot (-i)^n \prod_{p=1}^{n}e^{i\boldsymbol{\omega}_n[p]t_{l(p)}^{'}} dt_1^{'} ... dt_M^{'}
\end{equation}
\begin{equation}
    S^{(n)}(\boldsymbol{\omega}_n, \delta t) = \frac{1}{2^n}\sum_{\boldsymbol{m} \in \mathbb{Z}^{n+1}_+} S^{(n)}_{\boldsymbol{m}}(\boldsymbol{\omega}_n, \delta t)
\end{equation}
\begin{equation}
    U^{(n)}(t + \delta t, t) = \sum_{\left\{\boldsymbol{\omega}_n\right\}}e^{i\sum_{p=1}^{n}\boldsymbol{\omega}_n[p]t}S^{(n)}(\boldsymbol{\omega}_n, \delta t)
\end{equation}

de sorte que 

\begin{equation}
    U(t+\delta t, t) = \sum_{n=0}^{\infty}U^{(n)}(t+\delta t, t)
\end{equation}

\subsection{Ordre 0}
On calcule explicitement $U^{(0)}(t+\delta t, t)$, soit la branche la plus à gauche dans la figure 2 où la chaîne d'opérateurs ne contient que des $H_0$. Ici, $\boldsymbol{m} = [m_0]$ et $\left\{\boldsymbol{\omega}_0\right\}$ est vide. Dès lors,

\begin{equation*}
    S^{(0)}_{\boldsymbol{m}}(\boldsymbol{\omega}_0, \delta t) = \int_{0}^{\delta t}\int_{0}^{t_{m_0}^{'}}...\int_{0}^{t_2^{'}} (-iH_0)^{m_0}dt_1^{'}...dt_{m_0}^{'} = \frac{(-iH_0)^{m_0}}{m_0!}\int_{0}^{\delta t}...\int_{0}^{\delta t}dt_1^{'}...dt_{m_0}^{'} = \frac{(-iH_0\int_{0}^{\delta t}dt^{'})^{m_0}}{m_0!}
\end{equation*}
\begin{equation*}
    = \frac{(-iH_0\delta t)^{m_0}}{m_0!}
\end{equation*}

où on a utilisé le mettre truc qu'avec les $J_n$. Par la suite,

\begin{equation*}
    S^{(0)}(\boldsymbol{\omega}_0, \delta t) = \frac{1}{2^0}\sum_{\boldsymbol{m} \in \mathbb{Z}^{0+1}_{+}}S^{(0)}_{\boldsymbol{m}}(\boldsymbol{\omega}_0, \delta t) = \sum_{m_0=0}^{\infty}\frac{(-iH_0\delta t)^{m_0}}{m_0!} = e^{-iH_0\delta t}
\end{equation*}

Comme $\left\{\boldsymbol{\omega}_0\right\}$ est vide, on obtient finalement

\begin{equation*}
    U^{(0)}(t+\delta t, t) = e^{-iH_0\delta t}
\end{equation*}

Cela a du sens, car on a seulement des opérateurs indépendants du temps. Donc, on s'attend à ce que l'opérateur d'évolution soit de la même forme que (1.3), ce qui est le cas. De manière équivalente, on peut aussi faire le précédent calcul avec la définition de $H_0$. Ce résultat est évident, car $H_0$ est diagonal dans sa base d'états propres.

\begin{equation*}
    S^{(0)}(\boldsymbol{\omega}_0, \delta t) = \sum_{m_0 = 0}^{\infty}\frac{(-iH_0\delta t)^{m_0}}{m_0!} = \sum_{m_0 = 0}^{\infty}\frac{(-i\sum_{k}\lambda_k \ket{k}\bra{k}\delta t)^{m_0}}{m_0!} = \sum_{k}\sum_{m_0 = 0}^{\infty}\frac{(-i\lambda_k\delta t)^{m_0}}{m_0!}\ket{k}\bra{k} = \sum_{k}e^{-i\lambda_k\delta t}\ket{k}\bra{k}
\end{equation*}


\subsection{Ordre 1}
On s'attarde ici aux branches avec un seul $V(t)$ à quelque part dans la séquence d'opérateurs, ce qu'on calcule par $ U^{(1)}(t+\delta t, t)$. On sait que $\boldsymbol{m} = [m_1, m_0]$ et que $\left\{\boldsymbol{\omega}_1\right\} = \left\{[+\omega], [-\omega]\right\}$. Ainsi, 

\begin{equation*}
    S^{(1)}_{\boldsymbol{m}}(\boldsymbol{\omega}_1, \delta t) = -i\int_{0}^{\delta t}\int_{0}^{t_M^{'}}... \int_{0}^{t_2^{'}}(-iH_0)^{m_1}X(-iH_0)^{m_0} e^{i\boldsymbol{\omega}_1[1]t^{'}_{l(1)}}dt_1^{'} ... dt_M^{'}
\end{equation*}
\begin{equation*}
    = -i\int_{0}^{\delta t}\int_{0}^{t_M^{'}}... \int_{0}^{t_2^{'}}(-iH_0)^{m_1}X(-iH_0)^{m_0} e^{i\boldsymbol{\omega}_1[1]t^{'}_{m_0+1}}dt_1^{'} ... dt_M^{'}
\end{equation*}

Sachant dans ce cas que $M = m_0 + m_1 + 1 \implies m_0 + 1 = M-m_1$,

\begin{equation*}
    S^{(1)}_{\boldsymbol{m}}(\boldsymbol{\omega}_1, \delta t) = -i (-iH_0)^{m_1}X
    \int_{0}^{\delta t}\int_{0}^{t_M^{'}}... \int_{0}^{t_2^{'}}(-iH_0)^{m_0} e^{i\boldsymbol{\omega}_1[1]t^{'}_{M-m_1}}dt_1^{'} ... dt_M^{'}
\end{equation*}
\begin{equation*}
    = -i (-iH_0)^{m_1}X
    \int_{0}^{\delta t}dt_M^{'} ... \int_{0}^{t^{'}_{M-m_1+1}}e^{i\boldsymbol{\omega}_1[1]t^{'}_{M-m_1}} dt_{M-m_1}^{'}\int_{0}^{t^{'}_{M-m_1}}... \int_{0}^{t_2^{'}}(-iH_0)^{m_0} dt_1^{'} ... dt_{M - m_1 - 1}^{'}
\end{equation*}

De là,

\begin{equation*}
    S^{(1)}(\boldsymbol{\omega}_1, \delta t) = 
\end{equation*}
\begin{equation*}
    \frac{1}{2} \sum_{m_1 = 0}^{\infty}\sum_{m_0 = 0}^{\infty}\left(-i (-iH_0)^{m_1}X
    \int_{0}^{\delta t}dt_M^{'} ... \int_{0}^{t^{'}_{M-m_1+1}}e^{i\boldsymbol{\omega}_1[1]t^{'}_{M-m_1}} dt_{M-m_1}^{'}\int_{0}^{t^{'}_{M-m_1}}... \int_{0}^{t_2^{'}}(-iH_0)^{m_0} dt_1^{'} ... dt_{M - m_1 - 1}^{'}\right)
\end{equation*}
\begin{equation*}
    = \frac{-i}{2}\sum_{m_1 = 0}^{\infty}(-iH_0)^{m_1}X\int_{0}^{\delta t}dt^{'}_M ... \int_{0}^{t^{'}_{M-m_1+1}}e^{i\boldsymbol{\omega}_1[1]t^{'}_{M-m_1}} dt_{M-m_1}^{'}\sum_{m_0 = 0}^{\infty}\int_{0}^{t^{'}_{M-m_1}}... \int_{0}^{t_2^{'}}(-iH_0)^{m_0} dt_1^{'} ... dt_{M - m_1 - 1}^{'}
\end{equation*}

On a déjà rencontré à l'ordre 0 ce qui se trouve à droite de la somme infinie sur $m_0$. Il y a en effet $m_0$ intégrales à la droite de cette somme. Alors, on écrit

\begin{equation*}
    S^{(1)}(\boldsymbol{\omega}_1, \delta t) = \frac{-i}{2}\sum_{m_1 = 0}^{\infty}(-iH_0)^{m_1}X\int_{0}^{\delta t}dt^{'}_M ... \int_{0}^{t^{'}_{M-m_1+1}}e^{i\boldsymbol{\omega}_1[1]t^{'}_{M-m_1}} dt_{M-m_1}^{'}S^{(0)}(\boldsymbol{\omega}_0, t^{'}_{M-m_1})
\end{equation*}
\begin{equation*}
    = \frac{-i}{2}\sum_{m_1 = 0}^{\infty}(-iH_0)^{m_1}X\int_{0}^{\delta t}dt^{'}_M ... \int_{0}^{t^{'}_{M-m_1+1}}e^{i\boldsymbol{\omega}_1[1]t^{'}_{M-m_1}} dt_{M-m_1}^{'}\sum_{k^\text{'}}e^{-i\lambda_{k^\text{'}}t^{'}_{M-m_1}}\ket{k^\text{'}}\bra{k^\text{'}}
\end{equation*}
\begin{equation}
    = \frac{-i}{2}\sum_{k^\text{'}}\sum_{m_1 = 0}^{\infty}(-iH_0)^{m_1}X\int_{0}^{\delta t}dt^{'}_M ... \int_{0}^{t^{'}_{M-m_1+1}}e^{i(-\lambda_{k^{'}}+\boldsymbol{\omega}_1[1])t^{'}_{M-m_1}} dt_{M-m_1}^{'}\ket{k^\text{'}}\bra{k^\text{'}}
\end{equation}

On souhaite simplifier davantage en retrouvant encore l'équation d'ordre 0 pour y faire la même manipulation avec les $H_0$ restants. Cependant, l'ordre des intégrales imbriquées nous ne permet de faire cela directement. Pour y remédier, on se rappelle des démarches pour $J_2$ où on pouvait l'écrire de deux manières différentes, soit en intégrant "horizontalement" ou "verticalement" (voir figure 1 (A) et (B)). Ainsi, on pouvait inverser l'ordre d'intégration avec de nouvelles bornes appropriées. Cette façon de faire se généralise pour $J_n$ mais tient aussi lorsqu'on intègre des fonctions au lieu de matrices. Pour les intégrales imbriquées de (2.11), on trouverait pour un tel changement

\begin{equation*}
    S^{(1)}(\boldsymbol{\omega}_1, \delta t) = \frac{-i}{2}\sum_{k^\text{'}}\sum_{m_1 = 0}^{\infty}(-iH_0)^{m_1}X \int_{0}^{\delta t}e^{i(-\lambda_{k^{'}}+\boldsymbol{\omega}_1[1])t^{'}_{M-m_1}} dt_{M-m_1}^{'} \int_{t^{'}_{M-m_1}}^{\delta t} dt^{'}_{M-m_1+1} ... \int_{t^{'}_{M-1}}^{\delta t} dt^{'}_{M}  \ket{k^\text{'}}\bra{k^\text{'}}
\end{equation*}

À droite de l'unique intégrale avec l'exponentielle, il y a exactement $m_1$ intégrales. Alors, 

\begin{equation*}
    S^{(1)}(\boldsymbol{\omega}_1, \delta t) = \frac{-i}{2}\sum_{k^\text{'}} \int_{0}^{\delta t}e^{i(-\lambda_{k^{'}}+\boldsymbol{\omega}_1[1])t^{'}_{M-m_1}} dt_{M-m_1}^{'} \sum_{m_1 = 0}^{\infty}\int_{t^{'}_{M-m_1}}^{\delta t} dt^{'}_{M-m_1+1} ... \int_{t^{'}_{M-1}}^{\delta t} dt^{'}_{M}  (-iH_0)^{m_1}X\ket{k^\text{'}}\bra{k^\text{'}}
\end{equation*}

Encore une fois, par les démarches pour $J_n$, on peut dire

\begin{equation*}
    S^{(1)}(\boldsymbol{\omega}_1, \delta t) = \frac{-i}{2}\sum_{k^\text{'}} \int_{0}^{\delta t}e^{i(-\lambda_{k^{'}}+\boldsymbol{\omega}_1[1])t^{'}_{M-m_1}} dt_{M-m_1}^{'} \sum_{m_1 = 0}^{\infty}\frac{1}{m_1!}\left(\int_{t^{'}_{M-m_1}}^{\delta t} dt^{'}_{M-m_1+1}\right)^{m_1}(-iH_0)^{m_1}X\ket{k^\text{'}}\bra{k^\text{'}}
\end{equation*}
\begin{equation*}
    = \frac{-i}{2}\sum_{k^\text{'}} \int_{0}^{\delta t}e^{i(-\lambda_{k^{'}}+\boldsymbol{\omega}_1[1])t^{'}_{M-m_1}} dt_{M-m_1}^{'} \sum_{m_1 = 0}^{\infty}\frac{\left(-iH_0(\delta t - t^{'}_{M-m_1})\right)^{m_1}}{m_1!}X\ket{k^\text{'}}\bra{k^\text{'}}
\end{equation*}
\begin{equation*}
    = \frac{-i}{2}\sum_{k^\text{'}} \int_{0}^{\delta t}e^{i(-\lambda_{k^{'}}+\boldsymbol{\omega}_1[1])t^{'}_{M-m_1}} dt_{M-m_1}^{'} e^{-iH_0(\delta t - t^{'}_{M-m_1})} X\ket{k^\text{'}}\bra{k^\text{'}}
\end{equation*}
\begin{equation*}
    = \frac{-i}{2}\sum_{k^\text{'}} \int_{0}^{\delta t}e^{i(-\lambda_{k^{'}}+\boldsymbol{\omega}_1[1])t^{'}_{M-m_1}} dt_{M-m_1}^{'} \sum_{k} e^{-i\lambda_k(\delta t -  t^{'}_{M-m_1})}\ket{k}\bra{k}X\ket{k^\text{'}}\bra{k^\text{'}}
\end{equation*}
\begin{equation*}
    = \frac{-i}{2}\sum_{k, k^\text{'}}\int_{0}^{\delta t}e^{i(\lambda_k -\lambda_{k^{'}}+\boldsymbol{\omega}_1[1])t^{'}_{M-m_1}} dt_{M-m_1}^{'} e^{-i\lambda_k\delta t} \bra{k}X\ket{k^\text{'}}\ket{k}\bra{k^\text{'}}
\end{equation*}
\begin{equation*}
    = \frac{-i}{2}\sum_{k, k^\text{'}}\frac{1}{i(\lambda_k - \lambda_{k^\text{'}} + \boldsymbol{\omega}_1[1])}\left(e^{i\delta t(\lambda_k - \lambda_{k^\text{'}} + \boldsymbol{\omega}_1[1])} - 1\right)e^{-i\lambda_k\delta t} \bra{k}X\ket{k^\text{'}}\ket{k}\bra{k^\text{'}}
\end{equation*}
\begin{equation*}
    = \frac{-i}{2}\sum_{k, k^\text{'}}\frac{1}{-i(\lambda_k - (\lambda_{k^\text{'}} - \boldsymbol{\omega}_1[1]))}\left(e^{-i\lambda_k\delta t} - e^{-i\delta t(\lambda_{k^\text{'}} - \boldsymbol{\omega}_1[1])} \right) \bra{k}X\ket{k^\text{'}}\ket{k}\bra{k^\text{'}}
\end{equation*}
\begin{equation*}
    = \frac{-i\delta t}{2}\sum_{k, k^\text{'}}\frac{i}{\lambda_k\delta t - (\lambda_{k^\text{'}} - \boldsymbol{\omega}_1[1])\delta t}\left(e^{-i\lambda_k\delta t} - e^{-i\delta t(\lambda_{k^\text{'}} - \boldsymbol{\omega}_1[1])} \right) \bra{k}X\ket{k^\text{'}}\ket{k}\bra{k^\text{'}}
\end{equation*}
\begin{equation}
    = \frac{-i\delta t}{2}\sum_{k, k^\text{'}}f(\lambda_k\delta t, (\lambda_{k^\text{'}}-\boldsymbol{\omega}_1[1])\delta t)\bra{k}X\ket{k^\text{'}}\ket{k}\bra{k^\text{'}}
\end{equation}

où on pose

\begin{equation}
    f(\lambda_k\delta t, (\lambda_{k^\text{'}}-\boldsymbol{\omega}_1[1])\delta t) = \frac{i}{\lambda_k\delta t - (\lambda_{k^\text{'}} - \boldsymbol{\omega}_1[1])\delta t}\left(e^{-i\lambda_k\delta t} - e^{-i\delta t(\lambda_{k^\text{'}} - \boldsymbol{\omega}_1[1])} \right) 
\end{equation}

Finalement, le calcul de $U^{(1)}(t + \delta t, t)$ se complète avec (2.9).

On remarque qu'autant pour $S^{(0)}(\boldsymbol{\omega}_0, \delta t)$ que $S^{(1)}(\boldsymbol{\omega}_1, \delta t)$, ils ne dépendent pas du temps actuel $t$ dans l'évolution, mais seulement de l'écart de temps $\delta t$. D'ailleurs, cette observation sera aussi vrai pour les ordres supérieurs. Ainsi, si l'évolution totale dure un temps $T = P\delta t$ pour un entier $P$ (donc qu'on découpe l'évolution en $P$ parties égales), alors on peut tous les évaluer simultanément. Grâce à cela, on pourrait calculer un opérateur d'évolution général $U((p+1)\delta t, p\delta t)$ puis simplement y substituer le $p$ approprié pour reconstruire l'entièreté de l'évolution.




